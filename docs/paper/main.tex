% pdflatex -s main.tex && bibtex main.aux && pdflatex -s main.tex -o main.pdf && open main.pdf
\documentclass[11pt]{article}


%-----------------------------------------------------------------------------%
% Fonts and symbols:
%-----------------------------------------------------------------------------%

\usepackage{mathpazo}
\usepackage{amsfonts}
\usepackage{amsmath}
\usepackage{amssymb}
\usepackage{latexsym}
\usepackage{mathtools}
\usepackage{listings}


%-----------------------------------------------------------------------------%
% Margins and page layout:
%-----------------------------------------------------------------------------%

\usepackage[margin=1in]{geometry}
\usepackage{hyperref}
\hypersetup{pdfpagemode=UseNone}
\frenchspacing

%-----------------------------------------------------------------------------%
% Theorem-like environments:
%-----------------------------------------------------------------------------%

\usepackage{amsthm}
\newtheorem{theorem}{Theorem}
\newtheorem{observation}{Observation}
\newtheorem{conjecture}{Conjecture}
\newtheorem{lemma}[theorem]{Lemma}
\newtheorem{prop}[theorem]{Proposition}
\newtheorem{cor}[theorem]{Corollary}
\theoremstyle{definition}
\newtheorem{definition}[theorem]{Definition}
\newtheorem{remark}[theorem]{Remark}
\newtheorem{example}[theorem]{Example}

%-----------------------------------------------------------------------------%
% Other packages:
%-----------------------------------------------------------------------------%

\usepackage{authblk}
\usepackage{tikz}
\usetikzlibrary{calc,positioning}
\usepackage{mdframed}

\mdfdefinestyle{figstyle}{
  linecolor=black!7,
  backgroundcolor=black!7,
  innertopmargin=10pt,
  innerleftmargin=25pt,
  innerrightmargin=25pt,
  innerbottommargin=10pt
}

\definecolor{White}{rgb}{1,1,1}
\definecolor{Black}{rgb}{0,0,0}
\definecolor{LightGray}{rgb}{.81,.81,.81}
\colorlet{ChannelColor}{LightGray}
\colorlet{ChannelTextColor}{Black}
\colorlet{ReadoutColor}{White}

\usepackage{xcolor}

\definecolor{codegreen}{rgb}{0,0.6,0}
\definecolor{codegray}{rgb}{0.5,0.5,0.5}
\definecolor{codepurple}{rgb}{0.58,0,0.82}
\definecolor{tqblue}{HTML}{08293d}
\definecolor{backcolour}{HTML}{fefdf5}

\lstdefinestyle{mystyle}{
    backgroundcolor=\color{backcolour},
    commentstyle=\color{codegreen},
    keywordstyle=\color{magenta},
    numberstyle=\tiny\color{codegray},
    stringstyle=\color{codepurple},
    basicstyle=\ttfamily\footnotesize\color{tqblue},
    breakatwhitespace=false,
    breaklines=true,
    postbreak=\mbox{\textcolor{magenta}{$\hookrightarrow$}\space},
    captionpos=b,
    keepspaces=true,
    numbers=left,
    numbersep=5pt,
    showspaces=false,
    showstringspaces=false,
    showtabs=false,
    tabsize=2
}

\lstset{style=mystyle}

\usepackage{booktabs}
\usepackage{soul}

%-----------------------------------------------------------------------------%
% Macros:
%-----------------------------------------------------------------------------%

\newcommand{\comment}[1]{\begin{quote}\sf 
    \textcolor{blue}{[#1]}\end{quote}}
\newcommand{\linecomment}[1]{\textcolor{blue}{\sf[#1]}}

\newcommand{\tinyspace}{\mspace{1mu}}
\newcommand{\microspace}{\mspace{0.5mu}}
\newcommand{\negsmallspace}{\mspace{-1.5mu}}
\newcommand{\op}[1]{\operatorname{#1}}
\newcommand{\tr}{\operatorname{Tr}}
\newcommand{\pt}{\operatorname{T}}
\newcommand{\POS}{\operatorname{POS}}
\newcommand{\PPT}{\operatorname{PPT}}
\newcommand{\SEP}{\operatorname{SEP}}
\newcommand{\rank}{\operatorname{rank}}
\renewcommand{\int}{\operatorname{int}}
\renewcommand{\det}{\operatorname{Det}}
\renewcommand{\vec}{\operatorname{vec}}
\newcommand{\fid}{\operatorname{F}}
\newcommand{\im}{\operatorname{im}}

\renewcommand{\t}{{\scriptscriptstyle\mathsf{T}}}

\newcommand{\abs}[1]{\lvert #1 \rvert}
\newcommand{\bigabs}[1]{\bigl\lvert #1 \bigr\rvert}
\newcommand{\Bigabs}[1]{\Bigl\lvert #1 \Bigr\rvert}
\newcommand{\biggabs}[1]{\biggl\lvert #1 \biggr\rvert}
\newcommand{\Biggabs}[1]{\Biggl\lvert #1 \Biggr\rvert}

\newcommand{\ip}[2]{\langle #1 , #2\rangle}
\newcommand{\bigip}[2]{\bigl\langle #1, #2 \bigr\rangle}
\newcommand{\Bigip}[2]{\Bigl\langle #1, #2 \Bigr\rangle}
\newcommand{\biggip}[2]{\biggl\langle #1, #2 \biggr\rangle}
\newcommand{\Biggip}[2]{\Biggl\langle #1, #2 \Biggr\rangle}

\newcommand{\ceil}[1]{\lceil #1 \rceil}
\newcommand{\bigceil}[1]{\bigl\lceil #1 \bigr\rceil}
\newcommand{\Bigceil}[1]{\Bigl\lceil #1 \Bigr\rceil}
\newcommand{\biggceil}[1]{\biggl\lceil #1 \biggr\rceil}
\newcommand{\Biggceil}[1]{\Biggl\lceil #1 \Biggr\rceil}

\newcommand{\floor}[1]{\lfloor #1 \rfloor}
\newcommand{\bigfloor}[1]{\bigl\lfloor #1 \bigr\rfloor}
\newcommand{\Bigfloor}[1]{\Bigl\lfloor #1 \Bigr\rfloor}
\newcommand{\biggfloor}[1]{\biggl\lfloor #1 \biggr\rfloor}
\newcommand{\Biggfloor}[1]{\Biggl\lfloor #1 \Biggr\rfloor}

\newcommand{\norm}[1]{\lVert\tinyspace #1 \tinyspace\rVert}
\newcommand{\bignorm}[1]{\bigl\lVert\tinyspace #1 \tinyspace\bigr\rVert}
\newcommand{\Bignorm}[1]{\Bigl\lVert\tinyspace #1 \tinyspace\Bigr\rVert}
\newcommand{\biggnorm}[1]{\biggl\lVert\tinyspace #1 \tinyspace\biggr\rVert}
\newcommand{\Biggnorm}[1]{\Biggl\lVert\tinyspace #1 \tinyspace\Biggr\rVert}

\newcommand{\bigtriplenorm}[1]{
  \bigl\lvert\!\microspace\bigl\lvert\!\microspace\bigl\lvert #1 
  \bigr\rvert\!\microspace\bigr\rvert\!\microspace\bigr\rvert}

\newcommand{\Bigtriplenorm}[1]{
  \Bigl\lvert\!\microspace\Bigl\lvert\!\microspace\Bigl\lvert #1 
  \Bigr\rvert\!\microspace\Bigr\rvert\!\microspace\Bigr\rvert}

\newcommand{\biggtriplenorm}[1]{
  \biggl\lvert\!\microspace\biggl\lvert\!\microspace\biggl\lvert #1 
  \biggr\rvert\!\microspace\biggr\rvert\!\microspace\biggr\rvert}

\newcommand{\Biggtriplenorm}[1]{
  \Biggl\lvert\!\microspace\Biggl\lvert\!\microspace\Biggl\lvert #1 
  \Biggr\rvert\!\microspace\Biggr\rvert\!\microspace\Biggr\rvert}

\newcommand{\triplenorm}[1]{
  \left\lvert\!\microspace\left\lvert\!\microspace\left\lvert #1 
  \right\rvert\!\microspace\right\rvert\!\microspace\right\rvert}
\def\iso{\cong}
\newcommand{\defeq}{\triangleq}

\newcommand{\ket}[1]{
  \lvert\microspace #1 \microspace \rangle}

\newcommand{\bigket}[1]{
  \bigl\lvert\microspace #1 \microspace \bigr\rangle}

\newcommand{\Bigket}[1]{
  \Bigl\lvert\microspace #1 \microspace \Bigr\rangle}

\newcommand{\biggket}[1]{
  \biggl\lvert\microspace #1 \microspace \biggr\rangle}

\newcommand{\Biggket}[1]{
  \Biggl\lvert\microspace #1 \microspace \Biggr\rangle}

\newcommand{\bra}[1]{
  \langle\microspace #1 \microspace \rvert}

\newcommand{\bigbra}[1]{
  \bigl\langle\microspace #1 \microspace \bigr\rvert}

\newcommand{\Bigbra}[1]{
  \Bigl\langle\microspace #1 \microspace \Bigr\rvert}

\newcommand{\biggbra}[1]{
  \biggl\langle\microspace #1 \microspace \biggr\rvert}

\newcommand{\Biggbra}[1]{
  \Biggl\langle\microspace #1 \microspace \Biggr\rvert}

\newcommand{\I}{\mathbb{1}}

\newcommand{\setft}[1]{\mathrm{#1}}
\newcommand{\Density}{\setft{D}}
\newcommand{\Pos}{\setft{Pos}}
\newcommand{\Unitary}{\setft{U}}
\newcommand{\Herm}{\setft{Herm}}
\newcommand{\Lin}{\setft{L}}

\newcommand{\complex}{\mathbb{C}}
\newcommand{\real}{\mathbb{R}}
\renewcommand{\natural}{\mathbb{N}}
\newcommand{\integer}{\mathbb{Z}}

\newcommand{\qustop}{\texttt{qustop}}

\newcommand{\poly}{\mathit{poly}}

\newenvironment{mylist}[1]{\begin{list}{}{
	\setlength{\leftmargin}{#1}
	\setlength{\rightmargin}{0mm}
	\setlength{\labelsep}{2mm}
	\setlength{\labelwidth}{8mm}
	\setlength{\itemsep}{0mm}}}
	{\end{list}}

\newenvironment{namedtheorem}[1]
	       {\begin{trivlist}\item {\bf #1.}\em}{\end{trivlist}}

\newcommand{\X}{\mathcal{X}}
\newcommand{\Y}{\mathcal{Y}}
\newcommand{\Z}{\mathcal{Z}}
\newcommand{\W}{\mathcal{W}}
\newcommand{\A}{\mathcal{A}}
\newcommand{\B}{\mathcal{B}}
\newcommand{\V}{\mathcal{V}}
\newcommand{\U}{\mathcal{U}}
\newcommand{\C}{\mathcal{C}}
\newcommand{\D}{\mathcal{D}}
\newcommand{\E}{\mathcal{E}}
\newcommand{\F}{\mathcal{F}}
\newcommand{\M}{\mathcal{M}}
\newcommand{\N}{\mathcal{N}}
\newcommand{\R}{\mathcal{R}}
\newcommand{\Q}{\mathcal{Q}}
\renewcommand{\P}{\mathcal{P}}
\renewcommand{\S}{\mathcal{S}}
\newcommand{\T}{\mathcal{T}}
\newcommand{\K}{\mathcal{K}}
\renewcommand{\L}{\mathcal{L}}

\newcommand{\yes}{\text{yes}}
\newcommand{\no}{\text{no}}
\newcommand{\opt}{\text{opt}}
\newcommand{\LOCC}{\text{LOCC}}

\newcommand{\class}[1]{\textup{#1}}
\newcommand{\reg}[1]{\textsf{#1}}

\DeclareFixedFont{\ttb}{T1}{txtt}{bx}{n}{10}
\DeclareFixedFont{\ttm}{T1}{txtt}{m}{n}{10}
\definecolor{deepblue}{rgb}{0,0,0.5}
\definecolor{deepred}{rgb}{0.6,0,0}
\definecolor{deepgreen}{rgb}{0,0.5,0}
\newcommand\cppstyle{\lstset{
language=C++,
basicstyle=\ttm,
otherkeywords={uint8_t, __m256i, size_t, ASSERT_TRUE, EXPECT_TRUE, TEST, BENCHMARK},
keywordstyle=\ttb\color{deepblue},
emphstyle=\ttb\color{deepblue},
stringstyle=\color{deepgreen},
commentstyle=\fontfamily{txtt}\selectfont\color{gray},
showstringspaces=false,
literate={*}{{\char42}}1
         {-}{{\char45}}1
}}
\lstnewenvironment{cpp}[1][]
{\cppstyle\lstset{#1}}{}
\newcommand\pythonstyle{\lstset{
language=python,
basicstyle=\ttm,
morekeywords={assert,as,echo},
keywordstyle=\ttb\color{deepblue},
emphstyle=\ttb\color{deepblue},
stringstyle=\color{deepgreen},
commentstyle=\fontfamily{txtt}\selectfont\color{gray},
showstringspaces=false,
literate={*}{{\char42}}1
         {-}{{\char45}}1
}}
\lstnewenvironment{python}[1][]
{\pythonstyle\lstset{#1}}{}

\makeatletter
\let\@fnsymbol\@arabic
\makeatother

%-----------------------------------------------------------------------------%
% Main document
%-----------------------------------------------------------------------------%

\begin{document}

\title{\LARGE\bf \qustop: Quantum state optimizer}

\author[1]{Vincent Russo}
  
\renewcommand\Affilfont{\normalsize\itshape}
\renewcommand\Authfont{\large}
\setlength{\affilsep}{6mm}
\renewcommand\Authand{\rule{10mm}{0mm}}

\maketitle

\begin{abstract}
    The \qustop\ package is an open-source Python module for studying various
    quantum state optimization scenarios including calculating optimal values
    for quantum state distinguishability, quantum state exclusion, and quantum
    state cloning.
\end{abstract}

%------------------------------------------------------------------------------%
\section{Introduction}
\label{sec:intro}
%------------------------------------------------------------------------------%

%------------------------------------------------------------------------------%
\section{Definitions}
\label{sec:definitions}
%------------------------------------------------------------------------------%

It holds that
\begin{equation}
    0 \leq \omega_t(\eta, \LOCC) \leq \omega_t(\eta, \SEP) \leq \omega_t(\eta, \PPT) \leq \omega_t(\eta, \POS) \leq 1
\end{equation}
for all $t \in [U, M]$.

It holds that
\begin{equation}
    0 \leq \omega_U(\eta, T) \leq \omega_M(\eta, T) \leq 1
\end{equation}
for all $T \in [\LOCC, \SEP, \PPT, \POS]$.

The four Bell states $\ket{\psi_0}, \ldots, \ket{\psi_3} \in \complex^2 \otimes
\complex^2$ are defined as

\begin{equation}
    \begin{aligned}
        \ket{\psi_0} = \frac{1}{\sqrt{2}} \left(\ket{00} + \ket{11} \right), &\quad
        \ket{\psi_1} = \frac{1}{\sqrt{2}} \left(\ket{01} + \ket{10} \right), \\
        \ket{\psi_2} = \frac{1}{\sqrt{2}} \left(\ket{01} - \ket{10} \right), &\quad
        \ket{\psi_3} = \frac{1}{\sqrt{2}} \left(\ket{00} - \ket{11} \right).
    \end{aligned}
\end{equation}


\subsection{Minimum-error distinguishability of four two-qubit states via PPT measurements}

\begin{theorem}\label{thm:min_error_four_state_two_qubit_ppt}
    Let $\X = \Y = \complex^2$. For any PPT measurement $\{ P_1, P_2, P_3, P_4
    \} \subset \PPT(\X : \Y)$, the success probability of correctly
    discriminating the states corresponding to the set 
	\begin{equation}
        \left\{ 
            \ket{\phi_n^{(1)}},
            \ket{\phi_n^{(2)}},
            \ket{\phi_m^{(3)}},
            \ket{\phi_m^{(4)}} 
        \right\} \subset 
        \left(\X \otimes \Y \right)
	\end{equation}
    assuming a uniform distribution $p_1 = p_2 = p_3 = p_4 = 1/4$, is at most
    \begin{equation}
        \frac{1}{2} \left(1 + \frac{n+m}{2}\right)
    \end{equation}
    for all $n \in [0, 1]$ and $m \in [0, 1]$.
\end{theorem}
\begin{proof}
	Define the Hermitian operator
	\begin{equation}
        H_{n,m} = \frac{1}{2}
        \begin{pmatrix}
            1+n & 0 & 0 & 0 \\
            0 & 1+m & 0 & 0 \\
            0 & 0 & 1+m & 0 \\
            0 & 0 & 0 & 1+n
        \end{pmatrix}
        \in \Herm(\X \otimes \Y).
	\end{equation}
	It holds that
	\begin{equation}
        \frac{1}{4}\tr(H_{n,m}) = \frac{1}{2} \left(1 + \frac{n+m}{2}\right).
	\end{equation}
    We now desire to show that 
    \begin{equation}
        \begin{aligned}
            H_{n,m} - \pt_{\X}\left(\phi_n^{(1)}\right) \in \Pos(\X \otimes\Y),
            &\quad
            H_{n,m} - \pt_{\X}\left(\phi_n^{(2)}\right) \in \Pos(\X \otimes\Y), \\
            H_{n,m} - \pt_{\X}\left(\phi_m^{(3)}\right) \in \Pos(\X \otimes\Y),
            &\quad
            H_{n,m} - \pt_{\X}\left(\phi_n^{(4)}\right) \in \Pos(\X \otimes\Y),
        \end{aligned}
    \end{equation}
    for all $n \in [0, 1]$ and $m \in [0, 1]$. Observe that
    \begin{equation}\label{eq:partial-transpose-states}
        \begin{aligned}
            \pt_{\X}\left(\phi_n^{(1)}\right) = 
            \begin{pmatrix}
                \alpha^2 & 0 & 0 & 0 \\
                0 & 0 & \alpha\beta & 0 \\
                0 & \alpha\beta & 0 & 0 \\
                0 & 0 & 0 & \beta^2
            \end{pmatrix}, &\quad 
            \pt_{\X}\left(\phi_n^{(2)}\right) = 
            \begin{pmatrix}
                \beta^2 & 0 & 0 & 0 \\
                0 & 0 & -\alpha\beta & 0 \\
                0 & -\alpha\beta & 0 & 0 \\
                0 & 0 & 0 & \alpha^2
            \end{pmatrix}, \\ 
            \pt_{\X}\left(\phi_m^{(3)}\right) = 
            \begin{pmatrix}
                0 & 0 & 0 & \gamma\delta \\
                0 & \gamma^2 & 0 & 0 \\
                0 & 0 & \delta^2 & 0 \\
                \gamma\delta & 0 & 0 & 0
            \end{pmatrix}, &\quad 
            \pt_{\X} \left(\phi_m^{(4)}\right) = 
            \begin{pmatrix}
                0 & 0 & 0 & -\gamma\delta \\
                0 & \delta^2 & 0 & 0 \\
                0 & 0 & \gamma^2 & 0 \\
                -\gamma\delta & 0 & 0 & 0
            \end{pmatrix}.
        \end{aligned}
    \end{equation}
    We can represent $H_{n,m}$ as
    \begin{equation}
        H_{n,m} = 
        \begin{pmatrix}
            \alpha^2 & 0 & 0 & 0 \\
            0 & \gamma^2 & 0 & 0 \\
            0 & 0 & \gamma^2 & 0 \\
            0 & 0 & 0 & \alpha^2
        \end{pmatrix}
    \end{equation}
    since it holds that
    \begin{equation}
        \alpha^2 = \left(\sqrt{\frac{1+n}{2}}\right)^2 = \frac{1 + n}{2} 
        \quad \text{and} \quad 
        \gamma^2 = \left(\sqrt{\frac{1+m}{2}}\right)^2 = \frac{1 + m}{2}.
    \end{equation}
    Consider $k=1$ and observe that
    \begin{equation}
        H_{n,m} - \pt_{\X}\left(\phi_n^{(1)}\right) = 
        \begin{pmatrix}
            \alpha^2 - \alpha^2 & 0 & 0 & 0 \\
            0 & \gamma^2 & -\alpha\beta & 0 \\
            0 & -\alpha\beta & \gamma^2 & 0 \\
            0 & 0 & 0 & \alpha^2 - \beta^2
        \end{pmatrix}
    \end{equation}
    which is positive semidefinite as $0 \leq \beta \leq \alpha \leq 1$ and $0
    \leq \delta \leq \gamma \leq 1$ for all $n \in [0,1]$ and $m \in [0,1]$. A
    similar observation can be made for $k = 2, 3, 4$
\end{proof}

For completeness, we provide the following code listing that makes use of
\qustop to encode the setting from
Theorem~\ref{thm:min_error_four_state_two_qubit_ppt}.

\begin{figure}[!htpb]
    \centering
    \lstinputlisting[language=Python]{code/four_state_two_qubit_ppt.py}
    \caption{}
    \label{fig:ppt_four_state_two_qubit}
\end{figure}


\subsubsection*{Example: Indistinguishable maximally entangled states}

Consider the following ensemble in $\complex^4 \otimes \complex^4$ defined as
\begin{equation}\label{eq:ydy_ensemble}
    \eta = \left\{ \left(\frac{1}{4}, \ket{\psi_0} \otimes \ket{\psi_0} \right), 
                  \left(\frac{1}{4}, \ket{\psi_2} \otimes \ket{\psi_1} \right),
                  \left(\frac{1}{4}, \ket{\psi_3} \otimes \ket{\psi_1} \right),
                  \left(\frac{1}{4}, \ket{\psi_1} \otimes \ket{\psi_1} \right) 
            \right\} \subset \complex^4 \otimes \complex^4.
\end{equation}
In~\cite{yu2012four}, these states were shown to not be perfectly
distinguishable via PPT measurements, although the optimal probability with
which one is able to distinguish via PPT measurements was left as an open
question. This question was resolved in~\cite{cosentino2013positive}, where it
was shown that $\omega_M(\eta, \PPT) = 7/8$.

In~\cite{cosentino2013positive} it was also shown that $\omega_U(\eta, \PPT) = 3/4$.

\begin{figure}[!htpb]
    \centering
    \lstinputlisting[language=Python]{code/indstinguishable_mes.py}
    \caption{Calculate the minimum-error and unambiguous probabilities of
    distinguishing the ensemble from Equation~\eqref{eq:ydy_ensemble} via PPT
    measurements and separable measurements.}
    \label{fig:ppt_ydy}
\end{figure}


\subsubsection*{Example: Entanglement cost of distinguishing Bell states}

One may ask whether the ability to distinguish a state can be improved by
making use of an auxiliary resource state
\begin{equation}
    \ket{\tau_{\epsilon}} = \sqrt{\frac{1+\epsilon}{2}}\ket{00} + \sqrt{\frac{1-\epsilon}{2}},
\end{equation}
for some $\epsilon \in [0, 1]$.


\bibliographystyle{alpha}
\bibliography{refs}

\end{document}

