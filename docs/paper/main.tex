\documentclass[nofootinbib,superscriptaddress,a4paper,twocolumn,longbibliography,floatfix,pra]{revtex4-2}
\usepackage[english]{babel}
\usepackage[utf8]{inputenc}

\usepackage{enumitem}

\usepackage{amsfonts}
\usepackage{amsthm}
\usepackage{mathtools}
\usepackage{physics}
\usepackage{xcolor}
\usepackage{graphicx}
\usepackage[left=23mm,right=13mm,top=35mm,columnsep=15pt]{geometry} 
\usepackage{adjustbox}
\usepackage{placeins}
\usepackage[T1]{fontenc}
\usepackage{lipsum}
\usepackage{csquotes}

\usepackage{listings}
\usepackage{xcolor}

\definecolor{codegreen}{rgb}{0,0.6,0}
\definecolor{codegray}{rgb}{0.5,0.5,0.5}
\definecolor{codepurple}{rgb}{0.58,0,0.82}
\definecolor{tqblue}{HTML}{08293d}
\definecolor{backcolour}{HTML}{fefdf5}

\lstdefinestyle{mystyle}{
    backgroundcolor=\color{backcolour},   
    commentstyle=\color{codegreen},
    keywordstyle=\color{magenta},
    numberstyle=\tiny\color{codegray},
    stringstyle=\color{codepurple},
    basicstyle=\ttfamily\footnotesize\color{tqblue},
    breakatwhitespace=false,         
    breaklines=true,
    postbreak=\mbox{\textcolor{magenta}{$\hookrightarrow$}\space},                 
    captionpos=b,                    
    keepspaces=true,                 
    numbers=left,                    
    numbersep=5pt,                  
    showspaces=false,                
    showstringspaces=false,
    showtabs=false,                  
    tabsize=2
}

\lstset{style=mystyle}

\usepackage{booktabs}
\usepackage{soul}

\usepackage{xspace}
\newcommand{\expect}[2]{\ensuremath{\langle #1 \rangle_{#2}}}

\setlength{\parindent}{0cm}

\usepackage[pdftex, pdftitle={Article}, pdfauthor={Author}]{hyperref} % For hyperlinks in the PDF
\usepackage{xcolor}
\hypersetup{
    colorlinks,
    linkcolor={red!50!black},
    citecolor={blue!50!black},
    urlcolor={blue!80!black}
}

%-----------------------------------------------------------------------------%
% Macros:
%-----------------------------------------------------------------------------%

\newcommand{\comment}[1]{\begin{quote}\sf 
    \textcolor{blue}{[#1]}\end{quote}}
\newcommand{\linecomment}[1]{\textcolor{blue}{\sf[#1]}}

\newcommand{\tinyspace}{\mspace{1mu}}
\newcommand{\microspace}{\mspace{0.5mu}}
\newcommand{\negsmallspace}{\mspace{-1.5mu}}
\newcommand{\pt}{\operatorname{T}}
\newcommand{\ppt}{\operatorname{PPT}}
\newcommand{\sep}{\operatorname{Sep}}
\renewcommand{\int}{\operatorname{int}}
\renewcommand{\det}{\operatorname{Det}}
\renewcommand{\vec}{\operatorname{vec}}
\newcommand{\fid}{\operatorname{F}}
\newcommand{\im}{\operatorname{im}}

\renewcommand{\tr}{\operatorname{Tr}}
\renewcommand{\t}{{\scriptscriptstyle\mathsf{T}}}

\renewcommand{\abs}[1]{\lvert #1 \rvert}
\newcommand{\bigabs}[1]{\bigl\lvert #1 \bigr\rvert}
\newcommand{\Bigabs}[1]{\Bigl\lvert #1 \Bigr\rvert}
\newcommand{\biggabs}[1]{\biggl\lvert #1 \biggr\rvert}
\newcommand{\Biggabs}[1]{\Biggl\lvert #1 \Biggr\rvert}

\renewcommand{\ip}[2]{\langle #1 , #2\rangle}
\newcommand{\bigip}[2]{\bigl\langle #1, #2 \bigr\rangle}
\newcommand{\Bigip}[2]{\Bigl\langle #1, #2 \Bigr\rangle}
\newcommand{\biggip}[2]{\biggl\langle #1, #2 \biggr\rangle}
\newcommand{\Biggip}[2]{\Biggl\langle #1, #2 \Biggr\rangle}

\newcommand{\ceil}[1]{\lceil #1 \rceil}
\newcommand{\bigceil}[1]{\bigl\lceil #1 \bigr\rceil}
\newcommand{\Bigceil}[1]{\Bigl\lceil #1 \Bigr\rceil}
\newcommand{\biggceil}[1]{\biggl\lceil #1 \biggr\rceil}
\newcommand{\Biggceil}[1]{\Biggl\lceil #1 \Biggr\rceil}

\newcommand{\floor}[1]{\lfloor #1 \rfloor}
\newcommand{\bigfloor}[1]{\bigl\lfloor #1 \bigr\rfloor}
\newcommand{\Bigfloor}[1]{\Bigl\lfloor #1 \Bigr\rfloor}
\newcommand{\biggfloor}[1]{\biggl\lfloor #1 \biggr\rfloor}
\newcommand{\Biggfloor}[1]{\Biggl\lfloor #1 \Biggr\rfloor}

\newcommand{\bignorm}[1]{\bigl\lVert\tinyspace #1 \tinyspace\bigr\rVert}
\newcommand{\Bignorm}[1]{\Bigl\lVert\tinyspace #1 \tinyspace\Bigr\rVert}
\newcommand{\biggnorm}[1]{\biggl\lVert\tinyspace #1 \tinyspace\biggr\rVert}
\newcommand{\Biggnorm}[1]{\Biggl\lVert\tinyspace #1 \tinyspace\Biggr\rVert}

\newcommand{\bigtriplenorm}[1]{
  \bigl\lvert\!\microspace\bigl\lvert\!\microspace\bigl\lvert #1 
  \bigr\rvert\!\microspace\bigr\rvert\!\microspace\bigr\rvert}

\newcommand{\Bigtriplenorm}[1]{
  \Bigl\lvert\!\microspace\Bigl\lvert\!\microspace\Bigl\lvert #1 
  \Bigr\rvert\!\microspace\Bigr\rvert\!\microspace\Bigr\rvert}

\newcommand{\biggtriplenorm}[1]{
  \biggl\lvert\!\microspace\biggl\lvert\!\microspace\biggl\lvert #1 
  \biggr\rvert\!\microspace\biggr\rvert\!\microspace\biggr\rvert}

\newcommand{\Biggtriplenorm}[1]{
  \Biggl\lvert\!\microspace\Biggl\lvert\!\microspace\Biggl\lvert #1 
  \Biggr\rvert\!\microspace\Biggr\rvert\!\microspace\Biggr\rvert}

\newcommand{\triplenorm}[1]{
  \left\lvert\!\microspace\left\lvert\!\microspace\left\lvert #1 
  \right\rvert\!\microspace\right\rvert\!\microspace\right\rvert}
\def\iso{\cong}
\newcommand{\defeq}{\triangleq}

\newcommand{\bigket}[1]{
  \bigl\lvert\microspace #1 \microspace \bigr\rangle}

\newcommand{\Bigket}[1]{
  \Bigl\lvert\microspace #1 \microspace \Bigr\rangle}

\newcommand{\biggket}[1]{
  \biggl\lvert\microspace #1 \microspace \biggr\rangle}

\newcommand{\Biggket}[1]{
  \Biggl\lvert\microspace #1 \microspace \Biggr\rangle}

\newcommand{\bigbra}[1]{
  \bigl\langle\microspace #1 \microspace \bigr\rvert}

\newcommand{\Bigbra}[1]{
  \Bigl\langle\microspace #1 \microspace \Bigr\rvert}

\newcommand{\biggbra}[1]{
  \biggl\langle\microspace #1 \microspace \biggr\rvert}

\newcommand{\Biggbra}[1]{
  \Biggl\langle\microspace #1 \microspace \Biggr\rvert}

\newcommand{\I}{\mathbb{I}}

\newcommand{\setft}[1]{\mathrm{#1}}
\newcommand{\Density}{\setft{D}}
\newcommand{\Pos}{\setft{Pos}}
\newcommand{\Unitary}{\setft{U}}
\newcommand{\Herm}{\setft{Herm}}
\newcommand{\Lin}{\setft{L}}

\newcommand{\complex}{\mathbb{C}}
\renewcommand{\natural}{\mathbb{N}}
\newcommand{\integer}{\mathbb{Z}}

\usepackage{FiraSans}
\newcommand{\poly}{\mathit{poly}}
\newcommand{\toqitofont}{%
	\fontfamily{FiraSans}%
	\selectfont}


\newcommand{\toqito}{ $|${\toqitofont toqito}$\rangle$\xspace}
\newcommand{\numpy}{\texttt{numpy}\xspace}
\newcommand{\tequila}{\textsc{toqito}}
\newenvironment{mylist}[1]{\begin{list}{}{
	\setlength{\leftmargin}{#1}
	\setlength{\rightmargin}{0mm}
	\setlength{\labelsep}{2mm}
	\setlength{\labelwidth}{8mm}
	\setlength{\itemsep}{0mm}}}
	{\end{list}}

\newenvironment{namedtheorem}[1]
	       {\begin{trivlist}\item {\bf #1.}\em}{\end{trivlist}}

\newcommand{\X}{\mathcal{X}}
\newcommand{\Y}{\mathcal{Y}}
\newcommand{\Z}{\mathcal{Z}}
\newcommand{\W}{\mathcal{W}}
\newcommand{\A}{\mathcal{A}}
\newcommand{\B}{\mathcal{B}}
\newcommand{\V}{\mathcal{V}}
\newcommand{\U}{\mathcal{U}}
\newcommand{\C}{\mathcal{C}}
\newcommand{\D}{\mathcal{D}}
\newcommand{\E}{\mathcal{E}}
\newcommand{\F}{\mathcal{F}}
\newcommand{\M}{\mathcal{M}}
\newcommand{\N}{\mathcal{N}}
\newcommand{\R}{\mathcal{R}}
\newcommand{\Q}{\mathcal{Q}}
\renewcommand{\P}{\mathcal{P}}
\renewcommand{\S}{\mathcal{S}}
\newcommand{\T}{\mathcal{T}}
\newcommand{\K}{\mathcal{K}}
\renewcommand{\L}{\mathcal{L}}

\newcommand{\yes}{\text{yes}}
\newcommand{\no}{\text{no}}

\newcommand{\class}[1]{\textup{#1}}
\newcommand{\reg}[1]{\textsf{#1}}

\def\BB84{\mathsf{BB84}}
\def\CHSH{\mathsf{CHSH}}
\def\MUB{\mathsf{MUB}}
\def\qustop{\texttt{qustop}}

\begin{document}
\title{\qustop: Quantum state optimizer}

\author{Vincent Russo}
\email[E-mail:]{vincentrusso1@gmail.com}
\affiliation{-}

\date{\today}

\begin{abstract}
\end{abstract}

\maketitle

%-----------------------------------------------------------------------------%
\section{Introduction}
\label{sec:introduction}
%-----------------------------------------------------------------------------%

Documentation and tutorials are available on the $\qustop$
\href{https://github.com/vprusso/toqito}{GitHub repository}~\cite{toqito}.


%-----------------------------------------------------------------------------%
\section{Fundamental objects}
\label{sec:fundemental_objects}
%-----------------------------------------------------------------------------%

%-----------------------------------------------------------------------------%
\subsection{Quantum states and ensembles}\label{sec:ensembles}
%-----------------------------------------------------------------------------%


%-----------------------------------------------------------------------------%
\section{Applications of qustop}
\label{sec:applications}
%-----------------------------------------------------------------------------%

%-----------------------------------------------------------------------------%
\subsection{Calculating optimal probabilities of distinguishing quantum states}\label{sec:opt_dist}
%-----------------------------------------------------------------------------%

%-----------------------------------------------------------------------------%
\subsection{Global distinguishability}\label{sec:global}
%-----------------------------------------------------------------------------%

%-----------------------------------------------------------------------------%
\subsection{PPT distinguishability}\label{sec:ppt}
%-----------------------------------------------------------------------------%

%-----------------------------------------------------------------------------%
\subsection{Separable distinguishability}\label{sec:separable}
%-----------------------------------------------------------------------------%

%-----------------------------------------------------------------------------%
\subsection{Calculating optimal probabilities of excluding quantum states}\label{sec:opt_exclude}
%-----------------------------------------------------------------------------%

%-----------------------------------------------------------------------------%
\subsection{Calculating optimal probabilities of cloning quantum states}\label{sec:opt_clone}
%-----------------------------------------------------------------------------%

\begin{figure}[!htpb]
    \centering
    \lstinputlisting[language=Python]{code/indstinguishable_mes.py}        
    \caption{Calculate the minimum-error probability of optimally
    distinguishing the states from Equation~\eqref{eq:ydy} via PPT
    measurements.}
    \label{fig:ppt_ydy}
\end{figure}


%-----------------------------------------------------------------------------%
\subsubsection{Quantum state discrimination via separable measurements}
\label{sec:}
%-----------------------------------------------------------------------------%


%-----------------------------------------------------------------------------%
\section{Conclusion}
%-----------------------------------------------------------------------------%


%-----------------------------------------------------------------------------%
\section{Acknowledgements}
%-----------------------------------------------------------------------------%

\bibliography{main}
\clearpage
\end{document}
